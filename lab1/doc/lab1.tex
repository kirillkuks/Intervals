\documentclass[a4paper,12pt]{article}

\usepackage[hidelinks]{hyperref}
\usepackage{amsmath}
\usepackage{mathtools}
\usepackage{shorttoc}
\usepackage{cmap}
\usepackage[T2A]{fontenc}
\usepackage[utf8]{inputenc}
\usepackage[english, russian]{babel}
\usepackage{xcolor}
\usepackage{graphicx}
\usepackage{float}
\graphicspath{{./img/}}

\definecolor{linkcolor}{HTML}{000000}
\definecolor{urlcolor}{HTML}{0085FF}
\hypersetup{pdfstartview=FitH,  linkcolor=linkcolor,urlcolor=urlcolor, colorlinks=true}

\DeclarePairedDelimiter{\floor}{\lfloor}{\rfloor}

\renewcommand*\contentsname{Содержание}

\newcommand{\plot}[3]{
    \begin{figure}[H]
        \begin{center}
            \includegraphics[scale=0.6]{#1}
            \caption{#2}
            \label{#3}
        \end{center}
    \end{figure}
}

\begin{document}
    \begin{titlepage}
	\begin{center}
		{\large Санкт-Петербургский политехнический университет\\Петра Великого\\}
	\end{center}
	
	\begin{center}
		{\large Физико-механический иститут}
	\end{center}
	
	
	\begin{center}
		{\large Кафедра «Прикладная математика»}
	\end{center}
	
	\vspace{8em}
	
	\begin{center}
		{\bfseries Отчёт по лабораторной работе №2 \\по дисциплине «Анализ данных с интервальной неопределённостью»}
	\end{center}
	
	\vspace{5em}
	
	\begin{flushleft}
		\hspace{16em}Выполнил студент:\\\hspace{16em}Куксенко Кирилл Сергеевич\\\hspace{16em}группа: 5040102/20201
		
		\vspace{2em}
		
		\hspace{16em}Проверил:\\\hspace{16em}к.ф.-м.н., доцент\\\hspace{16em}Баженов Александр Николаевич
		
	\end{flushleft}
	
	
	\vspace{6em}
	
	
	\begin{center}
		Санкт-Петербург\\2023 г.
	\end{center}	
	
\end{titlepage}
    \newpage

    \tableofcontents
    \listoffigures
    \newpage

    \section{Постановка задачи}
    \quad Имеется две вещественные выборки $ \overline{X_{1}}, \overline{X_{2}} $.
    Необходимо построить из них две интервальные выборки $ X_{1}, X_{2} $
    и найти такой вещественный коэффициент $ R $, что выборка $ X_{1} \cup R X_{2} $
    будет наиболее совместной в смысле индекса Жаккара.

    \section{Теория}
    \subsection{Индекс Жаккара}
    \quad Индекс Жаккара определяет степень совместности двух интервалов $ x, y $.
    \begin{equation}
        JK(x, y) = \frac{wid(x \land y)}{wid(x \lor y)}
        \label{e:simplejaccard}
    \end{equation}

    Здесь $ \land, \lor $ представляют собой операции взятия минимума и максимума по включению
    в полной арифметике Каухера.
    Формула \ref{e:simplejaccard} легко может быть обобщена на случай интервальной выборки $ X = \{x_i\}_{i=1}^{n} $.
    
    \begin{equation}
        JK(X) = \frac{wid(\land_{i=1,n}x_i)}{wid(\lor_{i=1,n}x_i)}
        \label{e:jaccard}
    \end{equation}

    Видно, что $ JK(X) \in [-1, 1] $. Для удобства перенормируем значение $ JK(X) $
    так, чтобы оно было в интервале $ [0, 1] $.

    \begin{equation}
        JK(X) = \frac{1}{2} + \frac{1}{2} JK(X)
    \end{equation}

    \subsection{Нахождение оптимального значения R}
    Для нахождения оптимально $ R $ необоходимо сначала найти верхнюю и нижнюю границы $ \underline{R}, \overline{R} $.

    \begin{equation}
        \underline{R} = \frac{\min_{i=1,n}\underline{x_{1i}}}{\max_{i=1,n}\overline{x_{2i}}}
    \end{equation}
    \begin{equation}
        \overline{R} = \frac{\max_{i=1,n}\overline{x_{1i}}}{\min_{i=1,n}\underline{x_{2i}}}
    \end{equation}

    Затем оптимальное значение $ R $ может быть найдено методом половинного деления.

    \section{Реализация}
    \quad Весь код написан на языке Python (версии 3.7.3).
    \href{https://github.com/kirillkuks/Intervals/tree/master/lab1}{Ссылка на GitHub с исходным кодом}.

    \section{Результаты}
    \quad Данные были взяты из файлов \textsl{data/dataset1/+0\_5V/+0\_5V\_85.txt} и \textsl{data/dataset/-0\_5V/-0\_5V\_6.txt}.
    Обынтерваливание было произведенно следующим образом.
    \begin{equation}
        x = [x_0 - \varepsilon, x_0 + \varepsilon], \varepsilon \in U(0.01, 0.05)
    \end{equation}
    где $ x_0 $ - точечное значение, $ U(0.01, 0.05) $ - равномерное распределение.
    
    Сначала посмотрим на исходные интервальные выборки $ X_1, X_2 $.
    \plot{X1X2}{Исходные интервальные выборки}{p:x1x2}

    Посчитаем индекс Жаккара обеих выборок. $ JK(X_1) = 0.734, JK(X_2) = 0.721 $.
    Верхняя и нижняя границы $ \underline{R} = -0.967, \overline{R} = -0.961 $.
    Найдем оптимальное значение $ R $
    (для наглядности на графике \ref{p:jaccard} изабражён более широкий интервал значений $ R $).
  
    \plot{Jaccard}{Зависимость индекса Жаккара от значения $R$}{p:jaccard}

    Оптимальное значние $ R $ оказалось равно $ R_{opt} = -0.963 $
    Посторим объединённыую выборку $ X_1 \cup R_{opt} X_2 $.

    \plot{X1RX2}{Объединённая выборка $ X_1 \cup R_{opt} X_2 $}{p:x1rx2}

    \section{Обсуждение}

\end{document}